\documentclass[../main.tex]{subfiles}
\graphicspath{{figures/}{../figures/}}

\begin{document}

\officialDemand{
  \textbf{致谢}
  \begin{officialDemandText}
    致谢应以简短的文字对课题研究与论文撰写过程中曾直接给予帮助的人员
    (例如指导教师、答疑教师及其他人员)表示的谢意,
    这不仅是一种礼貌,
    也是对他人劳动的尊重,
    是治学者应当遵循的学术规范。字数一般不应超过500字。
  \end{officialDemandText}
  \begin{officialDemandText}
    致谢另起一页编排。
    “致谢”二字居中,
    用三号黑体,
    字间空一格,
    上下各空一行。
    “致谢”二字下空一行打印内容,
    用小四号宋体。
  \end{officialDemandText}
}

感谢我的队友张探探,
张探探对硬件的了解和学习教会我很多硬件知识,
并直接帮助本文的写作与论述,
感谢他的研究与对本文的贡献。

感谢魏晋雁,孙清,赵蕾老师,
以及所有参加2021年电子设计大赛的队员们,
他们客服种种困难,
为我们的学习提供新的平台,新的思想,新的视野,
老师的坚持告诉我们工科务实的比赛的形式风格与价值,
感受到了真正的比赛是怎样促进学风,
感受到了单片机与电子器件控制这一广大领域。
同时,
坚持探索的参赛队员们教会我很多我不理解不关心的硬件知识,
让我理解了计算机硬件的魅力,
让我不在认为计算机硬件难,看不懂,
通过我们不断的了解,试验,控制器件感受到计算机的更宏观的思想构型,
也体验了硬件的分析,
彻底刷新了我们的比赛没有意义,
没有提高能力的困局。
虽然比赛的难度是极大的,
换一种表述,
他的含金量是极高的。
通过一批又一批的努力,
我们是可以取得进步提高的,
祝愿参赛者取得好的成绩。

同时感谢第一次让我们举办这次比赛的刘小东老师和其他同意参赛的领导,
刘小东老师的教育思想,
认定这种短期看不到的比赛是由价值的,
同事他对教育的理解值得我学习体会,
同事,
他主张的批判性阅读直接促使我能批判的接受这一类硬件的代码,
用已有的知识去试验探索,学习。

感谢陆伟老师,
他在代码风格上的坚持和对软件设计的理解帮助我构造其中的软件,
让我坚信写出美观的代码对整体系统的构建是有帮助的。

此外感谢雷向辰,刘小冬,冯珊,杨海忠,王慧丽,张健,
罗清君,张云,
马君,许文丽,
魏晋雁,孙清,
周延杰,张娟,李薇,
张天宇, 张磊,王瑞,刘通,李建廷,汤慧,
贾辰凌,潘安,
韩瑞,马晓梅%
老师在过去给我的大学学习的支持和帮助。




\end{document}
