\documentclass[../main.tex]{subfiles}
\graphicspath{{figures/}{../figures/}}

\begin{document}

\officialDemand{
	\textbf{正文}
	\begin{officialDemandText}
		正文是毕业论文的主体和核心部分,
		不同学科专业和不同形式的毕业论文可以有不同的写作方式。
		应当层次分明、数据可靠、文字简炼、说理透彻、推理严谨、立论正确。
		正文一般包括绪论(或序言)、正文主体和结论三部分。
		绪论部分主要论述论文的选题背景、国内外研究现状分析、研究目的和主要内容等。
		正文主体是对研究工作的详细表述,一般由标题、文字、图、表格和公式等部分组成。
		结论应概括说明所进行工作的情况和价值,
		分析其优点和特色,指出创新所在,
		并应指出其中存在的问题和今后的改进方向,
		结论要简单、明确,篇幅不宜过长。
		毕业论文(设计)的篇幅一般不少于8000字。
	\end{officialDemandText}

	\textbf{注释}
	\begin{officialDemandText}
		注释是对论文中所应用的名词术语的解释,
		或是对引文出处的说明,采用脚注形式。
	\end{officialDemandText}
}


\end{document}
